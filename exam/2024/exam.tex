% LTeX: language=fr
% Copyright © 2014, Loïc Grobol <loic.grobol@gmail.com>
%
% Permission is granted to Do What The Fuck You Want To
% with this document.
%
% See the WTF Public License, Version 2 as published by Sam Hocevar
% at http://wtfpl.net if you need more details.

\RequirePackage{xparse}
% Settings
\newcommand\myname{Loïc Grobol}
\newcommand\mylab{}
\newcommand\pdftitle{Mathématiques pour le TAL : partiel}
\newcommand\mymail{loic.grobol@gmail.com}
\newcommand\titlepagetitle{\pdftitle}
\newcommand\titlepagesubtitle{}
\newcommand\docdate{2024-04-12}  % chktex 8
\newcommand\conference{M1 Plurital}

%%%% Type de document %%%%
\documentclass[a4paper, 11pt]{exam}
\pagestyle{headandfoot}

\usepackage{parskip}
%


%%%% Packages %%%%%
\usepackage{fontspec}
	\directlua{
		luaotfload.add_fallback(
			"myfallback",
			{
				"NotoColorEmoji:mode=harf;",
				"NotoSans:mode=harf;",
				"DejaVuSans:mode=harf;",
			}
		)
	}
	\setmainfont[
		UprightFont={* Regular},
		ItalicFont={* Italic},
		BoldFont={* SemiBold},
		BoldItalicFont={* SemiBold Italic},
		RawFeature={fallback=myfallback;multiscript=auto;},
	]{Libertinus Serif}
	\setsansfont[
		RawFeature={fallback=myfallback;multiscript=auto;}
	]{Libertinus Sans}
	\setmonofont[
		Scale=MatchLowercase,
		RawFeature={fallback=myfallback;multiscript=auto;},
	]{Fira Mono}

\usepackage{polyglossia}
    \setmainlanguage{french}
	\setotherlanguage[variant=british]{english}

\usepackage[notbib]{tocbibind}
\usepackage{emptypage}

\usepackage{amsfonts,amssymb}
\usepackage{amsmath,amsthm}
\usepackage{mathtools}	% AMS Maths service pack
	\newtagform{brackets}{[}{]}	% Pour des lignes d'équation numérotées entre crochets
	\mathtoolsset{showonlyrefs, showmanualtags, mathic}	% affiche les tags manuels (\tag et \tag*) et corrige le kerning des maths inline dans un bloc italique voir la doc de mathtools
	\usetagform{brackets}	% Utilise le style de tags défini plus haut
\usepackage{lualatex-math}

\usepackage[
	math-style=french,
	warnings-off={
		mathtools-colon,
		mathtools-overbracket,
	},
]{unicode-math}
	\setmathfont{Libertinus Math}

\usepackage{newunicodechar}
	\newunicodechar{√}{\sqrt}

\usepackage{mleftright}
\usepackage{interval}
\usepackage[d]{esvect}

\usepackage{microtype}
\usepackage[punct-after]{fnpct}  % Better display of footnote next to punctuation

\usepackage[refpage, intoc, notocbasic]{nomencl}
	\makenomenclature


\usepackage{tabularx}
\usepackage{booktabs}
\usepackage{siunitx}
	\sisetup{
		detect-all,
		group-separator=\text{\,},
	}
	\DeclareSIUnit{\quantity}{\relax}
	\DeclareSIUnit{\words}{mots}
	\DeclareSIUnit{\sentences}{phrases}
	% Needed for italics and bold numbers in siunitx S-aligned columns
	\robustify\itshape
	\robustify\bfseries

\usepackage{multicol}
\usepackage{ccicons}
\usepackage{bookmark}
\usepackage{caption}
	\captionsetup{skip=1ex, labelformat=empty}
\usepackage{lua-ul}

\usepackage{pseudo}
	\pseudodefinestyle{fullwidth}{
		begin-tabular=\tabularx{\linewidth}{@{}
			r
			% Labels
			>{\pseudosetup}
			% Indent, font, ...
			X
			% Code (flexible)
			>{\leavevmode\small\color{black!60}}
			% Comment styling
			p{0.45\linewidth}
			% Comments (fixed)
			@{}
		},
		end-tabular = \endtabularx,
		setup-append = \pseudoeq
	}

% \usepackage[backgroundcolor=white, bordercolor=orange]{todonotes}
%     \NewDocumentCommand\addlink{ O{} m }{\todo[color=green!40, #1]{#2}}

% \usepackage{tikz}
% 	% Colour palette from [Paul Tol's technical note](https://personal.sron.nl/~pault/data/colourschemes.pdf) v3.1
% 	% Bright scheme
% 	\definecolor{sronbrightblue}{RGB}{68, 119, 170}
% 	\definecolor{sronbrightcyan}{RGB}{102, 204, 238}
% 	\definecolor{sronbrightgreen}{RGB}{34, 136, 51}
% 	\definecolor{sronbrightyellow}{RGB}{204, 187, 68}
% 	\definecolor{sronbrightred}{RGB}{238, 102, 119}
% 	\definecolor{sronbrightpurple}{RGB}{170, 51, 119}
% 	\definecolor{sronbrightgrey}{RGB}{187, 187, 187}

% 	\definecolor{sronmutedindigo}{RGB}{51,34,136}

% 	% And my favourite purple
% 	\definecolor{myfavouritepurple}{RGB}{113, 10, 186}

% 	\NewDocumentCommand{\textnode}{O{}mm}{\tikz[remember picture, baseline=(#2.base), inner sep=0pt]{\node[#1] (#2) {#3};}}
% 	\NewDocumentCommand{\mathnode}{O{}mm}{\tikz[remember picture, baseline=(#2.base), inner sep=0pt]{\node[#1] (#2) {\(\displaystyle #3\)};}}
% 	% Global style
% 	\tikzset{
% 		>=stealth,
% 	}
% 	% Beamer utilities, for easy figure import
% 	\tikzset{
% 		visible on/.style={},
% 		accented/.style={text=accent, thick},
% 	}

% 	% Misc utilities
% 	\tikzset{
% 		true scale/.style={scale=#1, every node/.style={transform shape}},
% 	}

% 	\usetikzlibrary{tikzmark}
% 	\usetikzlibrary{matrix, chains}
% 	\usetikzlibrary{shapes, shapes.geometric}
% 	\usetikzlibrary{decorations.pathreplacing}
% 	\usetikzlibrary{positioning, calc, intersections}
% 	\usetikzlibrary{fit}

% 	% TikZ externalisation
% 	\usetikzlibrary{external}
% 	% Create the `tikzpics/` folder if it does not exist
% 	\usepackage{shellesc}
% 	\ShellEscape{mkdir _tikzpics}
% 	% Only externalise pictures on demand
% 	\tikzset{
% 		external/export=false,
% 		external/prefix=_tikzpics/,
% 	}
% 	\tikzexternalize

% \usepackage{forest}
% 	\useforestlibrary{linguistics}

% \usepackage{minted}
% 	\usemintedstyle{lovelace}
% 	\setminted{autogobble, tabsize=4}

\usepackage[
	english=american,
	french=guillemets,
	autostyle=true,
]{csquotes}
	\renewcommand{\mkbegdispquote}[2]{\itshape\let\emph\textbf}
	\renewcommand{\mkenddispquote}[2]{%
		\hfill\mbox{\upshape#1#2}%
	}
	% Like `\foreignquote` but use the outside language's quotes not the inside's
	\NewDocumentCommand\quoteforeign{m m}{\enquote{\textlang{#1}{#2}}}
	% \renewcommand{\mkbegdispquote}[2]{%
	%     \tikz[remember picture, overlay]{\fill[lightgray] (pic cs:quotestart) rectangle (pic cs:quoteend);}%
	%     \tikzmark{quotestart}%
	% }
	% \renewcommand{\mkenddispquote}[2]{%
	%     #1#2%
	%     \hfill\tikzmark{quoteend}%
	% }


% Annoying package loading order, so far we have:
% digraph packages {hyperref -> biblatex; hyperref -> hyperxmp; cleveref -> hyperref; fixme -> hyperref; bookmark -> hyperref}

% \usepackage[
% 	style=authoryear-comp,
% 	mergedate=basic,
% 	dashed=false,
% 	backend=biber,
% 	isbn=false,
% 	maxcitenames=2,
% 	uniquelist=false,
% 	maxbibnames=8,
% 	backref=true,
% ]{biblatex}
% 	% No small caps in french bib
% 	\DefineBibliographyExtras{french}{\restorecommand\mkbibnamefamily}
	% \addbibresource{biblio.bib}

\usepackage{hyperxmp}  % Load before hyperref
% \usepackage[a-3u]{pdfx}
% \usepackage[unicode]{hyperref}  % Apparently already loaded, no idea by whom. I hate you, latex
\hypersetup{
	pdftitle={\pdftitle},
	pdfauthor={\myname},
	pdfcontactemail={\mymail},
	colorlinks=true,
	breaklinks=true,
	hypertexnames=true,
	pdfdisplaydoctitle,
	pdflang={fr},
	keeppdfinfo,
}
% \usepackage{bookmark}  % Load after hyperref 
% \usepackage{footnotebackref}

% \usepackage[status=draft]{fixme} % Load after hyperref
% \usepackage[status=final]{fixme} % Load after hyperref
	% \fxsetup{theme=color}
\usepackage{cleveref}  % Load after hyperref


%%%% License %%%%
\usepackage[type={CC}, modifier={by}, version={4.0}]{doclicense}

%%% Custom commands
% Maths
\NewDocumentCommand\transpose{m}{\prescript{\mathrm{t}}{}{#1}}

\DeclareMathOperator\Img{Im}
\DeclareMathOperator\Ker{Ker}
\DeclareMathOperator\FFNN{FFNN}
\DeclareMathOperator\LSTM{LSTM}
\NewDocumentCommand\bLSTM{}{\overleftarrow{\LSTM}}
\NewDocumentCommand\fLSTM{}{\overrightarrow{\LSTM}}
\DeclareMathOperator\GRU{GRU}
\NewDocumentCommand\bGRU{}{\overleftarrow{\GRU}}
\NewDocumentCommand\fGRU{}{\overrightarrow{\GRU}}
\DeclareMathOperator*\argmax{arg\,max}
\DeclareMathOperator*\softmax{softmax}

\DeclarePairedDelimiter\abs{\lvert}{\rvert}
\DeclarePairedDelimiter\brackets{[}{]}
\DeclarePairedDelimiterX\compset[2]{\lbrace}{\rbrace}{#1\,\delimsize|\,\mathopen{}#2}
\DeclarePairedDelimiterX\innprod[2]{\langle}{\rangle}{#1\,\delimsize|\,\mathopen{}#2}
\DeclarePairedDelimiter\discseg{⟦}{⟧}
\DeclarePairedDelimiter\ceil{⌈}{⌉}
\DeclarePairedDelimiter\floor{⌊}{⌋}
\DeclarePairedDelimiter\round{⌊}{⌉}

\NewDocumentCommand\fonct{ m m m m m }{
	\left|\begin{array}{rrl}
		#1 :  & #2 & ⟶ #3 \\
				& #4 & ⟼ #1\mleft(#4\mright) = #5
	\end{array}\right.
}

% Easy column vectors \vcord{a,b,c} ou \vcord[;]{a;b;c}
% Here be black magic
\ExplSyntaxOn
	\NewDocumentCommand{\vcord}{O{,}m}{\vector_main:nnnn{p}{\\}{#1}{#2}}
	\NewDocumentCommand{\tvcord}{O{,}m}{\vector_main:nnnn{psmall}{\\}{#1}{#2}}
	\seq_new:N\l__vector_arg_seq
	\cs_new_protected:Npn\vector_main:nnnn #1 #2 #3 #4{
		\seq_set_split:Nnn\l__vector_arg_seq{#3}{#4}
		\begin{#1matrix}
			\seq_use:Nnnn\l__vector_arg_seq{#2}{#2}{#2}
		\end{#1matrix}
	}

	\NewDocumentCommand{\concat}{sO{,}m}{
		\IfBooleanTF{#1}{
			\ensuremath{\brackets*{\concat_main:nnn{,~}{#2}{#3}}}
		}{
			\ensuremath{\brackets{\concat_main:nnn{,~}{#2}{#3}}}
		}
	}
	\seq_new:N\l__concat_arg_seq
	\cs_new_protected:Npn\concat_main:nnn #1 #2 #3 {
		\seq_set_split:Nnn\l__concat_arg_seq{#2}{#3}
		\seq_use:Nnnn\l__concat_arg_seq{#1}{#1}{#1}
	}
\ExplSyntaxOff
	
	
\let\card\abs

% Misc
% from https://tex.stackexchange.com/a/27099/8547
\makeatletter
\NewDocumentCommand\centerfloat{}{%
	\parindent \z@%
	\leftskip \z@ \@plus 1fil \@minus \textwidth%
	\rightskip\leftskip%
	\parfillskip \z@skip%
}
\makeatother

\NewDocumentCommand\definition{ o m }{\emph{#2}}

\title{\titlepagetitle}
\author{\myname}
\date{\docdate}

%%%% Commandes spécifiques au document %%%%

\begin{document}
\pdfbookmark[1]{Cover}{cover}
\maketitle

\section{Applications directes}

\begin{questions}
	\question Développer et réduire la formule logique \((\operatorname{non}\,(A \operatorname{ou}\,B))\operatorname{ou}\,\operatorname{FAUX}\).
	\question Simplifier \(\{0, 1, 2, 3, 6\} ∩ \{0, 2, 4, 6, 8\}\)
	\question Résoudre le système d'équations suivant en utilisant l'algorithme de Gauß-Jordan
		\begin{equation}
			\begin{cases}
				x + y + z = 1\\
				2x - y + z = 2\\
				-x + y + z = 0\\
			\end{cases}
		\end{equation}
\end{questions}

\section{Droites vectorielles}
On dit qu'un ensemble \(E⊂ℝ²\) est une \emph{droite vectorielle de \(ℝ²\)} s'il existe un vecteur \(u∈ℝ²\) différent de \(0\) tel que

\begin{equation}
    E = \compset{λ⋅u}{λ∈ℝ}
\end{equation}

\begin{questions}
    \question Montrer que l'ensemble \(\compset{(x,y)∈ℝ²}{y = 2x}\) est une droite vectorielle. Dessiner cet ensemble dans un repère du plan et faire une conjecture sur la nature géométrique des droites vectorielles.
    \question Montrer que pour tout \(a∈ℝ\), l'ensemble \(\compset{(x, ax)∈ℝ²}{x∈ℝ}\) est une droite vectorielle.
    \question Y a-t-il des droites vectorielles qui ne sont pas de cette forme ?
    \question Soit \(D\) et \(D'\), deux droites vectorielles telles que \(D≠D'\), montrer que \(D∩D'=\{(0,0)\}\).
\end{questions}


\section{Projection}
Soit une matrice \(M∈\mathcal{M}_{2,2}(ℝ)\)

\begin{itemize}
    \item On appelle \emph{image} de \(M\) l'ensemble \(\Img(M)=\compset{M×X}{X∈ℝ²}\). Autrement dit, un vecteur \(Y\) appartient à l'image \(M\) si et seulement s'il existe un vecteur \(X\) tel que \(Y=MX\).
    \item On appelle \emph{noyau} de \(M\) l'ensemble \(\Ker(M)=\compset{X∈ℝ²}{M×X=0}\). 
\end{itemize}


Dans la suite, on considère la matrice \(M\) suivante

\begin{equation}
    M =
        \begin{pmatrix}
            \frac{1}{2} & \frac{1}{2}\\[.5em]
            \frac{1}{2} & \frac{1}{2}
        \end{pmatrix}
\end{equation}

\begin{questions}
    \question\label{q|points} Calculer \(M×X\) pour chaque vecteur \(X\) dans l'ensemble \(\{(0, 0), (1, 0), (0, 1), (1, 1), (1, 3), (1, 2)\}\).
    \question Placer (sous forme de points) les vecteurs trouvés dans la question précédente dans un repère du plan. Faire une hypothèse sur la nature de \(\Img(M)\).
    \question Résoudre le système d'équations suivant et représenter l'ensemble de ses solutions sur le repère précédent.
        \begin{equation}
            \begin{cases}
                \frac{1}{2}x + \frac{1}{2}y = 0
                \frac{1}{2}x + \frac{1}{2}y = 0
            \end{cases}
        \end{equation}
    \question Déterminer les \(X∈ℝ²\) tels que \(MX=0\). En déduire que \(\Ker(M)\) est une droite vectorielle.
    \question Soit \(X=(x,y)∈\Img(M)\). Que peut-on dire de \(x\) et \(y\) ? En déduire que \(\Img(M)\) est une droite vectorielle.
    \question Sur le repère précédent, tracer les droites passant par les points \(X\) et \(MX\), pour les \(X\) de la question \ref{q|points}. Faire une conjecture sur ces droites et le noyau de \(M\).
	\question Montrer que \(M²=M\).
\end{questions}

\section{Projecteur}

On appelle \emph{matrice de projection} toute matrice \(M\) telle que \(M²=M\). Noter que \(M\) est alors nécessairement une matrice carrée, et que la matrice nulle et la matrice identité sont toujours des matrices de projection.

Dans la suite on considère une matrice de projection \(M\) non-nulle et non-identité, et on notera \(E\) l'ensemble des vecteurs \(X\) tels que \(MX\) soit défini.

\begin{questions}
	\question Soit \(X∈E\), calculer \(M(X-MX)\). À quel ensemble appartient donc \(X-MX\) ?
	\question\label{q|sum} Soit \(X∈E\), à quel ensemble appartient le vecteur \(MX\) ? En déduire que pour tout \(X∈E\), il existe \(Y∈\Ker(M)\) et \(Z∈\Img(M)\) tels que \(X=Y+Z\).
	
		On dit que les ensembles \(\Ker(M)\) et \(\Img(M)\) \emph{engendrent} \(E\) et on note \(\Ker(M)+\Img(M)=E\).

	\question Montrer que \(\Ker(M)∩\Img(M)=\{0\}\).
	
		On dit que les ensembles \(\Ker(M)\) et \(\Img(M)\) sont \emph{en somme directe}, et puisque d'après la question \ref{q|sum}, ils engendrent \(E\), on dit qu'ils sont \emph{supplémentaires dans \(E\)} et on note \(\Ker(M)⊕\Img(M)=E\).

	\question Soit $(Y, Y')∈\Ker(M)²$ et $(Z, Z')∈\Img(M)²$. Montrer que $Y+Y'∈\Ker(M)$ et $Z+Z'∈\Img(M)$. En déduire que si $Y+Z=Y'+Z'$, alors $Y=Y'$ et $Z=Z'$.
	\question D'après les questions précédentes, si $X∈E$, il existe un unique couple $(Y, Z)∈\Ker(M)×\Img(M)$ tel que $X=Y+Z$. Que vaut alors $MX$ ?
\end{questions}

\end{document}
